% http://truben.no/table/
%\setlength\extrarowheight{1pt}
\begin{center} 
    \begin{tabular}{|c|c|c|}
    \hline
	\textbf{Bit}   & \textbf{Symbol bitu} & \textbf{Grupa}                                                            \\ \hline \hline
	0     & M     & \minibox[c]{początek minuty (zawsze 0)}                                      \\ \hline \hline
	1-14  & --     & \minibox[c]{informacja pogodowa\\(Meteotime)}                                 \\ \hline \hline
    15    & R     & nieprawidłowe działanie nadajnika              \\ \hline
	16    & A1    & \minibox[c]{nadchodzący czas letni\\(przez godzinę przed ustaleniem)}         \\ \hline
    17    & Z1    & czas letni                                                       \\ \hline
    18    & Z2    & czas zimowy                                                     	 \\ \hline
	19    & A2    & \minibox[c]{nadchodząca sekunda przestępna\\(przez godzinę przed ustaleniem)} \\ \hline \hline
	20    & S     & \minibox[c]{początek sygnału czasu (zawsze 1)}                               \\ \hline
	21-27 & --     & \minibox[c]{minuty\\(pierwszy bit najmłodszy)}                                \\ \hline
    28    & P1    & bit parzystości minuty                                           \\ \hline \hline
	29-34 & --     & \minibox[c]{godziny\\(pierwszy bit najmłodszy)}                               \\ \hline
    35    & P2    & bit parzystości godziny                                          \\ \hline 
	36-41 & --     & \minibox[c]{dzień miesiąca\\(pierwszy bit najmłodszy)}                        \\ \hline 
	42-44 & --     & \minibox[c]{dzień tygodnia\\(pierwszy bit najmłodszy)}                        \\ \hline 
    45-49 & --     & miesiąc                                                          \\ \hline
	50-57 & --     & \minibox[c]{rok (dwie cyfry)}                                                \\ \hline
	58    & P3    & \minibox[c]{bit parzystości daty\\(bity 36-57)}                               \\ \hline \hline
	59    & --     & \minibox[c]{znacznik nowej minuty\\(brak modulacji)}                          \\ \hline
    \end{tabular}
\end{center}
