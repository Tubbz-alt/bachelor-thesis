\newpage
\thispagestyle{empty}

\vspace*{1.5cm}
{	
	\LARGE
	\textsc{\textbf{Streszczenie}}
}
\vspace*{0.5cm}

Praca dyplomowa obejmuje badania zagadnień przetwarzania sygnałów,
w~szczególności radiowych. Prezentowany projekt ma na celu zastosowanie
techniki programowalnego radia w~środowisku \gls{fpga}. Implementacja sprzętowa
to zaprojektowana przez autora płytka drukowana, w~której użyto układów
radiowych, zadaniem których jest wstępna obróbka oraz przetworzenie sygnału
z~postaci analogowej do cyfrowej. Implementacja programowa to projekt \gls{fpga}
w~języku opisu sprzętu Verilog, w~którym zapisano algorytm przetwarzający
sygnał wejściowy. Ostatnim etapem jest aplikacja \gls{pc} w~języku Python, która
interpretuje przesłane przez układ dane.

\vspace*{1.5cm}
{
	\LARGE
	\textsc{\textbf{Abstract}}
}
\vspace*{0.5cm}

Maecenas eleifend venenatis tempor. Etiam tincidunt risus purus. Nulla aliquam nunc ex, sit amet elementum massa pellentesque feugiat. Donec sed turpis ut tellus vulputate dignissim. Cras nibh velit, rhoncus nec sodales vitae, auctor vel enim. Donec auctor felis quis lacus mollis porttitor. Morbi eu nulla id lectus tristique tincidunt ut blandit arcu. Sed mattis finibus ipsum, sit amet mollis quam tristique a. Curabitur nisl orci, suscipit nec odio ut, volutpat laoreet turpis. Fusce nec interdum felis, non sollicitudin lectus. Fusce tristique, ipsum eu convallis volutpat, lacus massa feugiat mauris, ac tincidunt massa felis ut libero. Cras interdum, lorem eget viverra pellentesque, nisi arcu suscipit velit, et laoreet ipsum lectus nec ligula.

