\newpage
\thispagestyle{empty}

\vspace*{1.5cm}
{	
	\LARGE
	\textsc{\textbf{Streszczenie}}
}
\vspace*{0.5cm}

Praca dyplomowa obejmuje badania zagadnień przetwarzania sygnałów,
w~szczególności radiowych. Prezentowany projekt ma na celu zastosowanie
techniki programowalnego radia w~środowisku \gls{fpga}. Implementacja sprzętowa
to zaprojektowana przez autora płytka drukowana, w~której użyto układów
radiowych, zadaniem których jest wstępna obróbka oraz przetworzenie sygnału
z~postaci analogowej do cyfrowej. Implementacja programowa to projekt \gls{fpga}
w~języku opisu sprzętu Verilog, w~którym zapisano algorytm przetwarzający
sygnał wejściowy. Ostatnim etapem jest aplikacja \gls{pc} w~języku Python, która
interpretuje przesłane przez układ dane.

\vspace*{1.5cm}
{
	\LARGE
	\textsc{\textbf{Abstract}}
}
\vspace*{0.5cm}

The thesis covers research on issues of digital signal processing, especially
wireless and radio signals. The project has a~goal to apply software defined
radio technique in \gls{fpga} environment. Hardware implementation designed by
the author is a~printed circuit board using several integrated circuits
which aim is to preliminary processing of a signal from analog to digital form.
Software implementation is a~\gls{fpga} project in Verilog hardware description
language in which an angorithm of processing input signal is written. Finally,
an application in Python language reads data sent by \gls{pcb} to a~PC and
converts it into readable form.
