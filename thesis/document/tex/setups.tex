
\author {Łukasz Stolcman}
\title {Dekoder sygnału wzorca czasu DCF77 z~wykorzystaniem radia programowalnego i~FPGA}

\usepackage {graphicx}

\usepackage {amsmath}

\usepackage [unicode]{hyperref}
%\usepackage [unicode,hidelinks]{hyperref}

\makeatletter
\hypersetup {pdfauthor={\@author}}
\hypersetup {pdftitle={\@title}}
\makeatother

\usepackage {multirow}

\usepackage {minibox}

\usepackage {array}

\usepackage {moreverb}

\usepackage {caption}

%\usepackage {makeidx}

\usepackage {xcolor}

\usepackage {listingsutf8}

% Align caption to the left
\DeclareCaptionFormat{listing}{\parbox{\dimexpr\textwidth-2\fboxsep+.8pt\relax}{#1#2#3}}
\captionsetup[lstlisting]{format=listing} 


\widowpenalty = 10000 % nie pozostawia wdów na koncu strony
\clubpenalty = 10000 % nie pozostawia sierot
\brokenpenalty = 10000 % nie dzieli stron jeśli podział wyrazu
\sloppy % zakaz wydłużania linii (gdy nie można złożyć)

%\makeindex

% horizontal alignment of multiline footnotes
\usepackage{scrextend}
%\deffootnote[2em]{2em}{1em}{\textsuperscript{\thefootnotemark}}
\deffootnote[2em]{2em}{1em}{\makebox[1em][l]{\textsuperscript{\thefootnotemark}}}

% change image padding after caption
% http://tex.stackexchange.com/a/23315
\setlength{\belowcaptionskip}{-15pt}
% http://tex.stackexchange.com/a/23316
\setlength{\abovecaptionskip}{-7pt}

%\usepackage[section]{placeins}

% multiple footnotes at one point divided by comma
% http://tex.stackexchange.com/a/28467
\usepackage[multiple]{footmisc}


