%\newglossaryentry{}{name={}, description={(\textit{ang. }) --- }}

\newglossaryentry{gsm}{name={GSM}, description={(\textit{ang.} Global System for Mobile Communications) --- najpopularniejszy standard telefonii komórkowej drugiej generacji}}

\newglossaryentry{pcb}{name={PCB}, description={(\textit{ang.} Printed Circuit Board) --- płytka z izolacyjnego materiału na której są umieszczone układy elektroniczne oraz ścieżki łączące je elektrycznie}}

\newglossaryentry{asic}{name={ASIC}, description={(\textit{ang. Application Specific Integrated Circuit}) --- układ elektroniczny zaprojektowany do pełnienia z góry określonego działania}}

\newglossaryentry{arm}{name={ARM}, description={(\textit{ang. Advanced RISC Machine, dawniej Acorn RISC Machine}) --- rodzina procesorów produkowana przez firmę ARM Ltd., szeroko wykorzystywana obecnie w urządzeniach wbudowanych}}

\newglossaryentry{dsp}{name={DSP}, description={(\textit{ang. Digital Signal Processing, Digital Signal Processor}) --- przetwarzanie sygnałów (lub) procesor specjalizowany posiadający dodatkowe funkcje przydatne przetwarzaniu sygnałów, np. mnożenie z akumulacją}}

\newglossaryentry{cad}{name={CAD}, description={(\textit{ang. Computer Aided Design}) --- zastosowanie komputera w projektowaniu technicznym}}

\newglossaryentry{fpga}{name={FPGA}, description={(\textit{ang. Field Programmable Gate Array}) --- programowalny układ logiczny o dużym stopniu scalenia}}

\newglossaryentry{dcf77}{name={DCF77}, description={--- Niemiecki sygnał czasu nadawany na falach długich}}

\newglossaryentry{sdr}{name={SDR}, description={(\textit{ang. Software Defined Radio}) --- system radiowy w którym większosć tradycyjnych elementów odbiorczych (lub nadawczych) została zastąpiona oprogramowaniem}}

\newglossaryentry{itu}{name={ITU}, description={(\textit{ang. International Telecommunication Union}) --- organizacja ustanowiona przez ONZ w celu standaryzacji i regulowania rynku tele- i radiokomunikacyjnego}}

\newglossaryentry{bcd}{name={BCD}, description={(\textit{ang. Binary-Coded Decimal}) --- sposób zapisu cyfr liczby dziesiętnej polegający na użyciu jej 4 młodszych bitów w reprezentacji binarnej}}

\newglossaryentry{pc}{name={PC}, description={(\textit{ang. Personal Computer}) --- komputer osobisty}}

\newglossaryentry{dvbt}{name={DVB-T}, description={(\textit{ang. Digital Video Broadcasting -- Terrestrial}) --- standard cyfrowej telewizji naziemnej}}

\newglossaryentry{adc}{name={ADC}, description={(\textit{ang. Analog to Digital Converter}) --- przetwornik analogowo-cyfrowy}}

\newglossaryentry{bnc}{name={BNC}, description={(\textit{ang. Bayonet Neill-Concelman}) --- złącze stosowane z kablem koncentrycznym, często spotykane w aparaturze pomiarowej (np. oscyloskopach)}}

\newglossaryentry{smd}{name={SMD}, description={(\textit{ang. Surface-Mount Devices}) --- rodzaj elementów elektronicznych nie posiadających drutów połączeniowych a jedynie metalowe końcówki, które umożliwiają przylutowanie elementu na płytkę drukowaną bez potrzeby wiercenia otworów}}

\newglossaryentry{pll}{name={PLL}, description={(\textit{ang. Phase Locked Loop}) --- układ elektroniczny działający na zasadzie sprzężenia zwrotnego, zwykle służący do automatycznej regulacji częstotliwości}}

\newglossaryentry{fir}{name={FIR}, description={(\textit{ang. Finite Impulse Response}) --- filtr cyfrowy o skończonej odpowiedzi impulsowej}}

